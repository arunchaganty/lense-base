\documentclass{article} % For LaTeX2e
\usepackage{nips15submit_09,times}
\usepackage{hyperref}
\usepackage{url}
\usepackage{times}
\usepackage{url}
\usepackage{latexsym}
\usepackage{graphicx}
\usepackage{bbm}
\usepackage{mathrsfs}
\usepackage{times,latexsym,amsfonts,amssymb,amsmath,graphicx,url,bbm,rotating,datetime}
\usepackage{enumitem,multirow,hhline,stmaryrd,bussproofs,mathtools,siunitx}
\usepackage{hyperref}
\usepackage{scabby}
\usepackage{tikz}
%\documentstyle[nips14submit_09,times,art10]{article} % For LaTeX 2.09

\providecommand{\byt}{\hat{\by}}
\providecommand{\bys}{{\by^*}}
\providecommand{\yt}{\hat{y}}
\providecommand{\ys}{{y^*}}
\providecommand{\Regret}{\operatorname{Regret}}

%\input std-macros.tex


\title{Fake it until you make it:\\Asynchronous Bayesian Active Classification\\with Crowds}

\author{%
Keenon Werling\\
Department of Computer Science\\
Stanford University\\
\texttt{keenon@cs.stanford.edu} \\
\And
Arun Chaganty\\
Department of Computer Science\\
Stanford University\\
\texttt{chaganty@cs.stanford.edu} \\
\And
Percy Liang\\
Department of Computer Science\\
Stanford University\\
\texttt{liang@cs.stanford.edu} \\
\And
Chris Manning\\
Department of Computer Science\\
Stanford University\\
\texttt{manning@cs.stanford.edu} \\
}

% The \author macro works with any number of authors. There are two commands
% used to separate the names and addresses of multiple authors: \And and \AND.
%
% Using \And between authors leaves it to \LaTeX{} to determine where to break
% the lines. Using \AND forces a linebreak at that point. So, if \LaTeX{}
% puts 3 of 4 authors names on the first line, and the last on the second
% line, try using \AND instead of \And before the third author name.

\newcommand{\fix}{\marginpar{FIX}}
\newcommand{\new}{\marginpar{NEW}}

\nipsfinalcopy % Uncomment for camera-ready version

\begin{document}

\usetikzlibrary{positioning}

\maketitle

\begin{abstract}
% V1
% Recent work in real-time crowd-powered products has demonstrated the possibility of using human ``crowd workers'' to power live, low latency products that accomplish AI complete tasks. Human-only solutions remain expensive, and do not scale into production. We propose a \textit{active classification} approach to dramatically reduce the scaling cost of such systems, backed by a pool of unreliable crowd-workers. In order to achieve this, we investigate \textit{asynchronous active classification}, where multiple requests can be simultaneously ``in-flight.'' This paper analyzes the \textit{asynchronous requests problem} of how to behave optimally in the presence of asynchronous oracle queries that have not yet returned, and the \textit{optimal stopping problem} of when a classification is ``good enough'' in such a setting. Our solutions to these problems enable a system that shows dramatic improvement over the human-only, machine-only, and baseline human-machine hybrid alternatives in the cost-delay-accuracy tradeoff surface.
Recent work in crowd-powered products has demonstrated the potential of using ``crowd workers'' to power live, low latency systems that accomplish AI complete tasks. 
These systems suffer from high scaling cost, and the unreliability of workers.
We propose an \textit{active classification} approach to dramatically reduce the scaling cost and improve the accuracy of such systems.
In order to also achieve low latency, we investigate \textit{asynchronous} active classification, where multiple requests can be simultaneously ``in-flight.''
This leads to the twin challenges of how to behave optimally in the presence of asynchronous noisy oracle queries that have not yet returned, and when to return a classification that is ``good enough'' in such a setting.
We first reduce the problem of optimal asynchronous active classification to a Partial Monitoring game, by making use of Bayesian decision theory.
We show a bound on achievable regret in this setting, and demonstrate practical heuristics that approach this bound.
We also show adaptations for traditional Active Learning algorithms to our setting.
We show empirical demonstration of our proposed algorithms, which demonstrate dramatic improvement over the human-only, machine-only, and baseline human-machine hybrid alternatives in the cost-delay-accuracy tradeoff surface.
\end{abstract}



\section{TODO}

\subsection{Arun}

\begin{itemize}
  \item I don't like that "money" and "time" are explicit variables in the loss function, mainly because it seems too concrete for abstract math. We can choose an specific loss function at a later time dependent on those parameters. I might yet come around to this notation.
\end{itemize}


\section{Introduction}

Recent work has shown that it is possible to use real-time on-demand workers to power everything from AI-complete email clients~\cite{kokkalis2013emailvalet} to real-time activity surveillance and classification~\cite{lasecki2013real}.
These purely crowd-based solutions are prohibitively expensive at scale.
Powering the crowd-based email client \textit{EmailValet}~\cite{kokkalis2013emailvalet} for a single end user for a year costs over \$400.

These systems typically work by ``pooling'' on-demand workers from high latency job-posting platforms like Amazon Mechanical Turk or CrowdFlower on a website designed by the system architect~\cite{lasecki2011real}.
 The ``pooling'' process can take several minutes, but once in place the workers can be queried at very low latencies by pushing requests to their web-browsers.
 This pool of workers can demonstrate high rates of turnover, and unreliability amongst individual annotators.

Existing systems query this pool directly, allowing for annotator noise by incorporating consensus building systems like voting and chat.


Active classification~\cite{greiner2002learning}, a close sibling of active learning, is a setting in which a classifier is allowed to query for more information, at some cost, before turning in classifications.
 This active classifier is intended to reduce its need for costly, slow human labels over time by learning from past observations.
 We propose to adapt the active classification framework to the pooled-worker setting to query this pool more cheaply, accurately, and quickly, without sacrificing the advantages live of crowd-powered interfaces.

Previous work in online active learning (which is closely related to what we're proposing) has focused on multi-class classification (\cite{chu2011unbiased},~\cite{agarwal2013selective},~\cite{cheng2013feedback},~\cite{vzliobaite2011active},~\cite{helmbold1997some}).
 Multi-class classification is an insufficiently rich primitive to handle many of the tasks that crowd-workers are enabling in existing systems, like information extraction or object detection.
 Instead, we will build our platform around arbitrary log-linear markov network classification, where we assume it is possible to query workers for opinions on individual nodes.
 Thus each ``active classification'' in our proposed setting is instantiate with a markov network and involves using model priors trained on previously seen data to choose to query the worker-pool for opinions about nodes, and then returning a classification informed by those opinions.


This setting poses several distinct challenges that have not been sufficiently addressed in previous literature.
 We need to be sensitive to time delays, returning results at least as quickly as the pure-crowd baseline we intend to improve upon.
 We also need to be sensitive to inaccurate oracles.
 These two criteria, in the pooled-worker setting, means that we need an active classifier who is able to hide latencies of redundant queries by launching them \textit{asynchronously}.
 This leads to the two challenges we will address in this paper, which can both be clustered under \textit{optimal asynchronous behavior}: we need to be able to handle the decision to ask for another query or turn in existing results in the presence of ``in-flight requests,'' which can fail due to worker turnover, where our loss term is sensitive to time delay.
 We draw inspiration from work in Bayesian Active Learning to tackle these problems.




\section{Related Work}
\label{sec:related}

In order to properly situate our proposal in prior art there's several key similarities to draw between traditional Active Learning and Active Classification.
 Active Learning is primarily concerned with generating a (possibly changing) ordering over remaining unlabeled examples, such that if examples are labeled one at a time and the model is retrained at each step, the maximum accuracy is achieved for minimum labeling cost.
 This is traditionally done over a static pool of unlabeled examples.
 To the extent that we can think of the nodes of our CRF as (dependent) examples to be labeled, this work is applicable.
 

Our log-linear markov networks are arriving in an online streaming setting, and there has been a large body of work investigating the ``streaming'' setting (\cite{chu2011unbiased},~\cite{agarwal2013selective},~\cite{cheng2013feedback},~\cite{vzliobaite2011active},~\cite{helmbold1997some}), where the algorithm visits a data instance once, and can choose to either request a label or discard it.
 Our labeling decision in the asynchronous delay-sensitive setting will be much more complicated than the decisions made by these algorithms.


Active Learning in the context of PGMs has been previously explored from several angles.
 Active Learning for CRF sequence models was famously explored in the context of batch active learning, where the oracle is able to provide labels for an entire sequence at once, and the objective was to achieve maximum accuracy for minimum cost~\cite{settles2008analysis}.
 It's also been explored in the context of fully Bayesian networks where the oracle\cite{tong2000active} can draw samples conditioned on certain ``controllable'' values (inspired by applications in experiment design).
 

Active Learning has also been explored for specific structure-prediction tasks,~\cite{roth2006active} and~\cite{culotta2005reducing}, where humans perform top-K selection over model predictions.
 The systems fall back by stages to traditional no-assistance annotation if the top-K doesn't contain the any correct information.
 While this approach is effective when possible, it relies on the model to consistently produce the correct answer in a top-K for some very small $K$, so for large output spaces it breaks down.


Active Learning where partial labels can be elicited has been shown effective in the batch setting with many PGMs for the relation extraction task~\cite{angeli2014combining}.
 We will follow this work in requesting partial observations of our graphs.


There's been a line of work on Active Learning in the context of multiple noisy, expensive oracles (\cite{yan2011active},~\cite{donmez2008proactive},~\cite{golovin2010near}).
 This work tries to relax the traditional assumptions in Active Learning that the oracle is infallible and has no economic cost.
 Some of this work is directly motivated by applications to crowd-sourcing platforms that we investigate.


Finally Bayesian Active Learning (\cite{golovin2010near},~\cite{tong2000active}) allows us to incorporate a Bayesian prior over our data, and we'll use this as a foundation for our approach to solving the asynchronous behavior problem.


\todo{Game playing lit review}

\subsection{Pure crowd classification systems}

Flock~\cite{chengflock} is a system of that attempts to put the entire classifier construction process, including proposing and annotating features, into the crowd.


Legion AR~\cite{lasecki2013real} is a system for realtime activity labeling.
 It provide useful priors for the kind of worker quality and longevity we can expect from the crowd.
 It also provides an implemented system for generating a real-time pool of workers.

\section{Background}
\label{sec:background}

In this section, we review some of the building blocks of our framework.

\paragraph{Learning with measurements}



\section{Setup}
\label{sec:problem}

We will now formalize how our task is evaluated, modelled and executed. 
We consider the ``designer'' to be the person who implements our system: she must decide what the model is and define an appropriate cost function.
A ``user'' of the system is the person providing input that needs to be labeled. 
Finally, ``labelers'' in the system provide potentially noisy partial labels (or measurements) on inputs from the user. \todo{diagram}
% Note that the designer, user and labeler can all be the same person.

\paragraph{Measuring progress.}

First, let's look at how answers produced by the system are evaluated.
In the conventional supervised setting, a natural metric is the accuracy of our predictions relative to the true labels in the training data.
The system gets to see these true labels and is able to optimize an appropriate loss.

Formally, let $\bx\oft{1}, \dots, \bx\oft{t}, \dots, \bx\oft{T}$ be a sequence of inputs, 
let $(\by\oft{t})$ be their true labels,
and let $(\byt\oft{t})$ be the sequence of predictions made by the system.
We measure the performance of our system using the loss $\ell(\by\oft{t}, \byt\oft{t})$, and our objective is to minimize the loss on the dataset: $\sum_{t=1}^T \ell(\by\oft{t}, \byt\oft{t})$.

When constructing datasets from user interactions, the system does not get to observe $\by\oft{t}$ or $\ell(\by\oft{t}, \byt\oft{t})$.
We are only aware of the true labels if and when we query a human annotator: simply measuring performance incurs a cost for our system.
Let $\sigma\oft{t}$ be the measurements queried by the system, incurring a cost $C(\sigma\oft{t})$.
Our goal is to minimize the objective
\begin{align*}
  \sL &= \sum_{t=1}^T \ell(\by\oft{t}, \byt\oft{t}) + C(\sigma\oft{t}).
\end{align*}

\paragraph{Model}

We consider the family of conditional exponential models:
\begin{align*}
  p_\theta(y \given x) 
  &= \exp( \theta^\top \phi(x, y) - A(\theta; x)),
\end{align*}
where $A(\theta; x)$ is the conditional log-normalizer.
We assume the model has low treewidth or otherwise admits efficient marginal computation.

As mentioned earlier, we do not have any labeled examples $y$, but can ask for some measurement $\sigma \in \Sigma$ by asking the crowd. 
In order to incorporate this information, we assume the distribution of responses to be modeled as an exponential measurement model as in~\cite{liang09measurements}:
\begin{align*}
  p(\tau \given x, y, \sigma) 
  &\propto \exp \left( h_\sigma(\tau - \sigma(x,y)) \right),
\end{align*}
where $h_\sigma$ is a convex function.
Consider a simple model for noisy labeling, where a person returns the true label with a uniform probability of $1- \epsilon$, and guesses at random otherwise. In this case, 
\begin{align*}
  h_\sigma(u) &= \alpha \I(\tau \neq \sigma),
\end{align*}
where $\alpha$ is equal to the inverse sigmoid:$\alpha = \sigma\inv(1 - \epsilon/2)$.

Typical examples of measurements are partial labels.
We assume that incorporating these measurements can be done efficiently, i.e.\ marginals can be efficiently computed for the joint model,
\begin{align*}
  p_\theta(y \given x, \tau, \sigma) 
  &\propto \exp(\theta^\top \phi(x, y) + h_\sigma(\tau - \sigma(x,y))).
\end{align*}

\paragraph{Measurement selection}

The key computation we need to perform is to choose a measurement action.
Assume that $\Sigma = \{\sigma_0, \sigma_1, \dots, \sigma_n\}$ are the set of valid measurement operations one can perform. 
Furthermore, let $\sigma_0$ be a special operation that does not ask for any measurement: it returns the most probable  labeling.

Let $\ell(\by, \byt)$ be the loss incurred when labeling $\by$ versus $\byt$.
Let the utility of a particular measurement operation be:
\begin{align*}
U(\sigma)
&= \E_{p(\tau \given x, \sigma)} \left[
      \E_{p(\by \given \bx, \tau)}[\ell(\by, \byt(\tau))] \right] + C(\sigma),
\end{align*}
where $\byt(\tau) = \argmax_{\by} p(\by \given \bx, \tau)$ is the maximum likelihood labeling $C(\sigma)$ is the cost function.

Intuitively, we will only ever choose to measure something if the expected utility between labellings is more than the cost of executing the measurement.

\paragraph{Asynchronous requests}

For the system to be real-time, we need to dispatch multiple measurement queries at the same time. Thus, we must reason about what portfolio of measurements would be optimal.

Iterated expectations work.

\paragraph{Optimization}

On receiving a label, we optimize to train our loss.

Q: Should we do gradients when we don't see data, basically doing an unsupervised update?

\section{Problem definition}

In summary, we will receive a stream of queries, and a pool of humans.
 We will be asked to minimized our risk (aka expected loss) in the Bayesian Decision Theoretic sense, over some loss function $\sL(h_{\text{true}}, h, m, t)$ that takes as parameters the true state of the world $h_{\text{true}}$, our guess $h$, and the money $m$ and time $t$ spent arriving at our guess.  

\subsection{The Queries}

We receive a stream of queries.
 The world, in the context of a query, is composed of vector of $n$ random variables, $[X_1 \ldots X_n] \in \sX$, where $\sX$ is the input domain.
 Each variable $X_i$ is categorical with domain $D_i = [1 \ldots K_i]$, so the size of domain $|D_i| = K_i$, and $K_i$ need not equal $K_j$ if $i \neq j$.

Each query asks us to distinguish between a number of possible \textit{hypotheses} $h \in \sH$, where $\sH$ is the space of all possible assignments to the $X_i$'s, so $\sH = D_1 \times \ldots \times D_n$.
 Each hypothesis $h$ corresponds to a unique assignment to a set of query variables $[X_1 = x_1, \ldots, X_n = x_n]$. 

Each query is further given a markov network to describe the features of the data relating the $X_i$'s.
 The markov net consists of a set of features over the data $f \in \sF$.
 Here $\sF$ is the space of all possible features extracted over families of variables $X_i$.

The likelihood of a single joint assignment $\bX = \bx$, where $\bx$ corresponds to some $h \in \sH$, is the normalized soft max of the inner product of the weights of our model $w$ (not given in the query, but learned over time) and the feature values at $\bx$.


\[P(\bX = \bx) = \frac{1}{Z}exp(\sum_k w_k^Tf_k(\bx_{\{k\}}))\]
\[Z = \sum_{x\in\sX} exp(\sum_k w_k^Tf_k(\bx_{\{k\}}))\]

To provide compact representation for the rest of the paper, we will say that each query is provided a \textit{graph}, which is a tuple of variables and feature functions $G = ([X_1, \ldots, X_n], [f_1, \ldots, f_m])$.


\subsection{The Crowd Workers}

Our learner will be allowed to query humans for additional certainty about individual random variables (nodes) $X_i$.
 Humans will exist in a pool $\sP$.
 An individual human $o_i \in \sP$ (for ``oracle'', using the loosest possible definition of oracle) has a model for expected behavior.
Our learner will be allowed to query humans for additional certainty about individual random variables (nodes) $X_i$.
 Humans will exist in a pool $\sP$.
 An individual human $o_i \in \sP$ (for ``oracle'', using the loosest possible definition of oracle) has a model for expected behavior. Specifically, we model the $o_i$ expected delay to respond to a question, and the error function (i.e. response given the true state of the world $h_{\text{true}}$).

For this work we use a simplified model of error, shared uniformly across crowd workers.
 Previous work \cite{yan2011active} \cite{donmez2008proactive} \cite{golovin2010near} has shown that in an offline setting treating oracles uniformly leads to a loss, but in practice our pool is churning so quickly that we don't have time to learn accurate distinctions between workers.
 We leave a solution to this problem to future work.

When asked about variable $X_j$, human error is modeled as correct (returns $h_{\text{true}}(X_j)$) with probability $1-\epsilon$, and chosen uniformly at random from $D_j$ with probability $\epsilon$.
 We use the notation $Q(o_i, X_j)$ to denote the response from asking human $i$ about random variable $j$.

\begin{equation}
    Q(o_i, X_j) \sim
    \begin{cases}
       \text{uniformly drawn from } D_j = \{1 \ldots K_j\}, & \text{with probability}\ \epsilon \\
      h_{\text{true}}(X_j), & \text{otherwise}
    \end{cases}
 \end{equation}
 
This answer arrives after a delay.
 We model the amount of time the worker takes to answer with a gaussian, parameterized by $\mu$ and $\sigma$.
 We use the notation $\sD(o_i, X_j)$ to denote the delay in response when asking human $i$ about random variable $j$.

\[\sD(o_i, X_j) \sim \sN(\mu, \sigma^2)\]

This is an imperfect model, since it assigns some mass to a negative response time, which is impossible, but given a large $\mu$ and relatively small $\sigma$ the mass assigned to $\sD(o_i, X_j) < 0$ is negligible, and it otherwise accurately reflects human response times.

\subsection{The Loss Function}

Our goal is to choose our querying strategy to minimize an arbitrary loss function:
\[\sL(h_{\text{true}}, h, m, t)\]
$\sL$ takes as parameters the true state of the world $h_{\text{true}}$, our guess $h$, and the money $m$ and time $t$ spent arriving at our guess.
 By tweaking $\sL$ practitioners applying our framework can move to arbitrary points on the cost-delay-accuracy tradeoff surface.
 We deliberately make no assumptions about the form of $\sL$ while developing theory, and demonstrate several choices in our experiments.




% Fun story description!
%Eve has spent some time thinking about her labeling task and has constructed a model $p_{\theta}( \by \given \bx )$, the specifics of which are irrelevant at this point.
%At some time $t$, a user of her system requests a data point $\bx\oft{t}$ to be labeled. The user expects a particular labeling $\ys$, though Eve and her system are not aware of this.
%Instead, Eve returns the model's best prediction: $\byt = \argmax_{\by} p_{\theta\oft{t}}(\by \given \bx\oft{t})$. 
%The user upon seeing this labeling implicitly (or explicitly) penalizes Eve for any inaccuracies using the loss function $\ell(\byt, \bys)$, either by not paying for her service or not using it in the future. 
%Eve unfortunately never gets to see this feedback.


\section{Asynchronous Active Classification}
\label{sec:async}

We will allow multiple queries to be simultaneously ``in-flight'' to different workers in $\sP$, to hide the latency associated with getting multiple human opinions.
 To justify the added complexity of our algorithm and analysis, if $q$ is the number of queries sent to oracles, runtime for a single active MAP estimate will scale as $O(\text{max}(1, \frac{q}{|\sP|}))$.
 If we restricted ourselves to sequential querying, runtime would scale as $O(q)$.
 Human classification requests can take \textit{seconds}, so if restricted to sequential querying our $t$ delay term in our loss $\sL$ will often force the algorithm to turn in suboptimal answers, even if there were idle annotators in $\sP$ who could have been contributing.

To model asynchrony, we will use an ``event-driven'' policy to make decisions for our active classifier.
 At a high level, the algorithm will ``wake-up'' whenever the state is changed, and will see a \textit{snapshot} of the state $s \in \sS$ at that instant, including information about request currently in-flight, and make a decision about the next action $a \in \sA$ to perform immediately according to its \textit{asynchronous policy} $\pi \in \sS \times \sA$.
 This action is then performed, and the game player gets another chance to move immediately.
 The game player can choose as one of its moves to go back to sleep until the next event.

We now formalize $\sS$ and $\sA$, and then provide an example run of a classification below to clarify exactly what all this machinery is doing.


The state space $\sS = \sG \times \sQ \times \sP$ consists of the query graph $G \in \sG$, and set of queries $Q \in \sQ$, and the current state of the annotator pool.
 Each individual query $q \in Q$ contains 3 pieces of state: (human $o_i$, variable $X_j$, status $S$).
 Status can be any of 
\[S \in [(\text{IN-FLIGHT}, t \in \sR^+), (\text{SUCCESS}, v \in D_j), \text{FAILED}]\]

The action space $\sA$ consists of asking for a query on a given node $X_i$ to a given human $o_i$, waiting until woken up again, or returning the answer immediately, so 
\[\sA = [\text{QUERY} \times \sP \times \sX , \text{WAIT}, \text{RETURN}]\]
This action is chosen according to a policy $\pi \in \sA$.

\subsection{Simple Example Run}

\todo{This can be dramatically compressed with proper formatting}

Here we show a simple run for a query with two variables, $X_1$ and $X_2$. The model is uncertain about the value of $X_1$, but relatively confident about $X_2$, so it will ask for two simultaneously requests for $X_1$, wait for them both to return, then turn in an answer.

\begin{enumerate}
\item Request graph $G = ([X_1, X_2], [f_1])$ received.
 Our current pool contains two annotators, $\sP = [o_1, o_2]$.
 We initialize our state $s \in \sS$ as $s = (G, [], [o_1, o_2])$. At this point the markov network diagram for our state looks as follows:

%%% TIKZ
\begin{center}
  \begin{tikzpicture}[
  auto=left,
  every node/.style={circle, draw=black, outer sep=+0pt, line width=0.5mm},
  node distance=1cm]

  \node              (n1) {$X_1$};
  \node[right=of n1] (n2) {$X_2$};

  \path[thick] (n1) edge (n2);
  
\end{tikzpicture}
\end{center}
%%% TIKZ

\item We then wake up the game player, and pass it $s = (G, [], [o_1, o_2])$.
 It returns us the action $a = (\text{QUERY}, o_1, X_1)$, the state is updated by adding a query $q_1 = (o_1, X_1, (\text{IN-FLIGHT}, 0ms))$ to $Q$.
 Our new state is $s = (G, [q_1], [o_1, o_2])$.
 Now the network looks as follows:

%%% TIKZ
\begin{center}
  \begin{tikzpicture}[
  auto=left,
  every node/.style={circle, draw=black, outer sep=+0pt, line width=0.5mm},
  node distance=1cm]

  \node              (n1) {$X_1$};
  \node[right=of n1] (n2) {$X_2$};
  \node[below=of n1] (n3) {$q_1=$?};

  \path[thick] (n1) edge (n2);
  \path[thick] (n1) edge (n3);
  
\end{tikzpicture}
\end{center}
%%% TIKZ

\item We then \textit{immediately} wake up the game player again, to ask what to do next.
 We pass it $s = (G, [q_1], [o_1, o_2])$, and it returns $a = (\text{QUERY}, o_2, X_1)$, so we update the state by adding $q_2 = (o_2, X_1, (\text{IN-FLIGHT}, 0ms))$ to $Q$.
 Our new state is $s = (G, [q_1, q_2], [o_1, o_2])$.
 Now the network looks as follows:

%%% TIKZ
\begin{center}
  \begin{tikzpicture}[
  auto=left,
  every node/.style={circle, draw=black, outer sep=+0pt, line width=0.5mm},
  node distance=1cm]

  \node              (n1) {$X_1$};
  \node[right=of n1] (n2) {$X_2$};
  \node[below left=of n1] (n3) {$q_1=$?};
  \node[below right=of n1] (n4) {$q_2=$?};

  \path[thick] (n1) edge (n2);
  \path[thick] (n1) edge (n3);
  \path[thick] (n1) edge (n4);
  
\end{tikzpicture}
\end{center}
%%% TIKZ

\item We then \textit{immediately} wake up the game player again, to ask what to do next.
 We pass it $s = (G, [q_1, q_2], [o_1, o_2])$, and it returns $a = (\text{WAIT})$.
 This means we need to put the game player to sleep until the results of our requests return.

\item $q_2$ returns, so the state of $q_2 = (o_2, X_1, (\text{SUCCESS}, 1))$.
 We pass the game player $s = (G, [q_1, q_2], [o_1, o_2])$, and it returns $a = (\text{WAIT})$, so we continue to sleep.

\item $q_1$ returns, and updates its state to $q_1 = (o_1, X_1, (\text{SUCCESS}, 1))$.
 We pass the game player $s = (G, [q_1, q_2], [o_1, o_2])$, and it returns $a = (\text{RETURN})$, so we take the MAP estimate over the current markov network, and return.
 Our graph looks like this in the final state:

%%% TIKZ
\begin{center}
  \begin{tikzpicture}[
  auto=left,
  every node/.style={circle, draw=black, outer sep=+0pt, line width=0.5mm},
  node distance=1cm]

  \node              (n1) {$X_1$};
  \node[right=of n1] (n2) {$X_2$};
  \node[below left=of n1] (n3) {$q_1=1$};
  \node[below right=of n1] (n4) {$q_2=1$};

  \path[thick] (n1) edge (n2);
  \path[thick] (n1) edge (n3);
  \path[thick] (n1) edge (n4);
  
\end{tikzpicture}
\end{center}
%%% TIKZ

\end{enumerate}

The important things to notice in this example are steps 3-5.
 These are points where the policy needs to look at an incomplete state, where requests have been made but not returned results yet, and decide how to proceed in the face of this potential future value already invested.
 We now proceed to explore how that is done.

\section{Asynchronous Active Learning}

In addition to the Partial Monitoring Game, our active classification task can be cast as an Active Learning task, which allows us to unlock the rich Active Learning literature.
 The traditional Active Learning algorithms are meant to operate in a game space where the branching factor is so prohibitively large as to necessitate single-step lookahead approximations.
 These approximations make an action based on a greedy metric of ``gain'' (defined differently by each algorithm) per unit ``cost'' (defined by the user).
 Formally, we choose an action $a$ based on observed values $\bx$ (presumed perfect), and cost of actions $c(a)$:

\[a^* = \argmax_{a \in \sA} \Delta_{\text{Alg}}(a | \bx) / c(a)\]

These heuristic solutions will need to be adapted to the asynchronous and Bayesian setting to be able to assign the correct expected value to in-flight requests.
 We will use a similar trick throughout.
 We are interested in greedily maximizing our loss, so we set $c(a) = \Delta E_{h \sim p(h)}[\sL(h_{\text{final}}, h, m, t)]$ where $h_{\text{final}}$ is the MAP assignment after turning in our guess.
 Then we can take an expectation over our beliefs about $\bx$ after all currently in-flight queries have returned.

\[a^* = \argmax_{a \in \sA} E_{\bx}[ \Delta_{\text{Alg}}(a | \bx) / \Delta \sL(a)]\]

\subsection{Uncertainty Sampling}

We pick the variable to query that our model has the \textit{least confidence} in.
 \todo{details}

\subsection{Value of Information}
\subsection{Information Gain}

\subsection{Query By Committee}

The idea here is to minimize the number of consistent hypothesis $h \in \sH$ over time by choosing to learn information that maximally reduces the hypothesis space.
 This is approximated by looking for maximum disagreement amongst a set of learners trained on random subsets of the data.
 \todo{details}

\subsection{Expected Gradient Length}

Here we choose the items that would impose the greatest update to the model if we knew its label.
 \todo{details}

\subsection{Information Density}

\todo{details}


\section{Partial Monitoring Game}
\label{sec:partial}

To cast this problem as a Partial Monitoring Game, we must first make sure the outcome space $\sS$ is finite, which requires that we discretize time into epochs of length $t_{\text{epoch}}$, and impose a time limit on the game, $t_{\text{max}}$, resulting in a finite game space $\sS_{\text{discrete}}$.

\todo{details on how Partial Monitoring Games work}

We can define the entries of our loss matrix $L$ with respect to our loss function $\sL(h_{\text{final}}, h, m, t)$.
 The language $\Sigma$ used to populate our observability matrix $H$ is defined as the state space of the game $\sS_{\text{discrete}}$.

\todo{do math... borrow bounds from Bartok et. al, show Pareto optimal policy}

Our goal is to pick a policy $\pi^*$ to minimize the loss in the terminal state of the algorithm $s_{\text{final}}$.
 Since our loss term depends on the true state of the world $\theta^* = (h_{\text{true}}, t, m)$, of which only $t$ and $m$ are ever known, we can't do this directly.
 However, we have a distribution over our beliefs $p(h_{\text{true}})$ at every step (and consequently the state of the true state of the world $\theta^*$, since $t$ and $m$ are known), so we instead minimize \textit{risk}, which is defined with respect to a particular state of the world $\tilde{\theta}$ as
\[E_{\Theta \sim p(\theta)}[Loss(\Theta || \tilde{\theta})] = \int_{\theta} Loss(\theta || \tilde{\theta})p(\theta)d\theta\]

We can formulate our goal in concrete terms as follows:
\begin{equation*}
\begin{aligned}
& \underset{\pi}{\text{minimize}}
& & E_{\pi}[E_{h \sim p(h)}[\sL(h_{\text{final}}, h, m, t)]] \\
%& \text{subject to}
%& & a \in \sA
\end{aligned}
\end{equation*}

This reduces to the Optimal Decision Tree problem, which is NP-hard.


\section{Experiments}
\label{sec:experiments}

We run our active classifier on the above policies, and compare against machine-only and human-only baselines, as well as a baseline policy that uses a threshold-version of uncertainty sampling without any effort to minimize loss.
 We report results on several datasets: CoNLL NER, Stanford sentiment classification, and \todo{more}.

For these results, we use a simulated crowd, using the model that with probability $\epsilon = 0.3$ the humans choose uniformly at random.

\subsection{CoNLL Evaluation}

The CoNLL NER dataset is composed of \todo{details}.\\
\todo{I realize that the ``token accuracy'' metric is \textbf{not} how NER is measured, in process of fixing}

\begin{tabular}{ | l | r | r | r | }
    \hline
    System & Avg classification time/token & Avg requests/token & Accuracy \\ \hline
    Human 1-query baseline & 345 ms & 1.0 & 75.67 \\ \hline
    Human 3-query baseline & 791 ms & 3.0 & 84.98 \\ \hline
    Offline baseline & n/a & n/a & 94.43 \\ \hline
    \textbf{Uncertainty threshold} & \textbf{326 ms} & \textbf{0.353} & \textbf{97.03} \\
    \hline
\end{tabular}

\subsection{Stanford Sentiment Dataset Evaluation}

\todo{write}


\section{Conclusion}
\label{sec:conclusion}

We demonstrate a practical system to navigate the cost-delay-accuracy tradeoff surface by \textit{asynchronous active classification}.
 This system highlights the potential value of making the crowd available to machine-learning classifiers at test time, and we hope inspires further research in this promising direction.

Much of the research effort associated with demonstrating this system was ``plumbing,'' writing and debugging large pieces of logistical code for marshaling and managing the humans and machine learning algorithms in an asynchronous fashion.
 We have published an easily extensible and customizable base for further work in asynchronous active classification, which we call L.E.N.S.E. (Learning from Expensive, Noisy, Slow Experts) at \url{github.com/lense-project/lense-base}, that we hope proves useful.



% I don't think we need this just yet.
%\subsubsection*{Acknowledgments}
%
%\todo{Arun: fill in your sponsors etc}

\subsubsection*{References}

\bibliographystyle{plain}
\bibliography{ref,all}

\end{document}
