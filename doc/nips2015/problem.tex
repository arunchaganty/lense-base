\section{Setup}
\label{sec:problem}

We will now formalize how our task is evaluated, modelled and executed. 
We consider the ``designer'' to be the person who implements our system: she must decide what the model is and define an appropriate cost function.
A ``user'' of the system is the person providing input that needs to be labeled. 
Finally, ``labelers'' in the system provide potentially noisy partial labels (or measurements) on inputs from the user. \todo{diagram}
% Note that the designer, user and labeler can all be the same person.

\paragraph{Measuring progress.}

First, let's look at how answers produced by the system are evaluated.
In the conventional supervised setting, a natural metric is the accuracy of our predictions relative to the true labels in the training data.
The system gets to see these true labels and is able to optimize an appropriate loss.

Formally, let $\bx\oft{1}, \dots, \bx\oft{t}, \dots, \bx\oft{T}$ be a sequence of inputs, 
let $(\by\oft{t})$ be their true labels,
and let $(\byt\oft{t})$ be the sequence of predictions made by the system.
We measure the performance of our system using the loss $\ell(\by\oft{t}, \byt\oft{t})$, and our objective is to minimize the loss on the dataset: $\sum_{t=1}^T \ell(\by\oft{t}, \byt\oft{t})$.

When constructing datasets from user interactions, the system does not get to observe $\by\oft{t}$ or $\ell(\by\oft{t}, \byt\oft{t})$.
We are only aware of the true labels if and when we query a human annotator: simply measuring performance incurs a cost for our system.
Let $\sigma\oft{t}$ be the measurements queried by the system, incurring a cost $C(\sigma\oft{t})$.
Our goal is to minimize the objective
\begin{align*}
  \sL &= \sum_{t=1}^T \ell(\by\oft{t}, \byt\oft{t}) + C(\sigma\oft{t}).
\end{align*}

\paragraph{Model}

We consider the family of conditional exponential models:
\begin{align*}
  p_\theta(y \given x) 
  &= \exp( \theta^\top \phi(x, y) - A(\theta; x)),
\end{align*}
where $A(\theta; x)$ is the conditional log-normalizer.
We assume the model has low treewidth or otherwise admits efficient marginal computation.

As mentioned earlier, we do not have any labeled examples $y$, but can ask for some measurement $\sigma \in \Sigma$ by asking the crowd. 
In order to incorporate this information, we assume the distribution of responses to be modeled as an exponential measurement model as in~\cite{liang09measurements}:
\begin{align*}
  p(\tau \given x, y, \sigma) 
  &\propto \exp \left( h_\sigma(\tau - \sigma(x,y)) \right),
\end{align*}
where $h_\sigma$ is a convex function.
Consider a simple model for noisy labeling, where a person returns the true label with a uniform probability of $1- \epsilon$, and guesses at random otherwise. In this case, 
\begin{align*}
  h_\sigma(u) &= \alpha \I(\tau \neq \sigma),
\end{align*}
where $\alpha$ is equal to the inverse sigmoid:$\alpha = \sigma\inv(1 - \epsilon/2)$.

Typical examples of measurements are partial labels.
We assume that incorporating these measurements can be done efficiently, i.e.\ marginals can be efficiently computed for the joint model,
\begin{align*}
  p_\theta(y \given x, \tau, \sigma) 
  &\propto \exp(\theta^\top \phi(x, y) + h_\sigma(\tau - \sigma(x,y))).
\end{align*}

\paragraph{Measurement selection}

The key computation we need to perform is to choose a measurement action.
Assume that $\Sigma = \{\sigma_0, \sigma_1, \dots, \sigma_n\}$ are the set of valid measurement operations one can perform. 
Furthermore, let $\sigma_0$ be a special operation that does not ask for any measurement: it returns the most probable  labeling.

Let $\ell(\by, \byt)$ be the loss incurred when labeling $\by$ versus $\byt$.
Let the utility of a particular measurement operation be:
\begin{align*}
U(\sigma)
&= \E_{p(\tau \given x, \sigma)} \left[
      \E_{p(\by \given \bx, \tau)}[\ell(\by, \byt(\tau))] \right] + C(\sigma),
\end{align*}
where $\byt(\tau) = \argmax_{\by} p(\by \given \bx, \tau)$ is the maximum likelihood labeling $C(\sigma)$ is the cost function.

Intuitively, we will only ever choose to measure something if the expected utility between labellings is more than the cost of executing the measurement.

\paragraph{Asynchronous requests}

For the system to be real-time, we need to dispatch multiple measurement queries at the same time. Thus, we must reason about what portfolio of measurements would be optimal.

Iterated expectations work.

\paragraph{Optimization}

On receiving a label, we optimize to train our loss.

Q: Should we do gradients when we don't see data, basically doing an unsupervised update?

\section{Problem definition}

In summary, we will receive a stream of queries, and a pool of humans.
 We will be asked to minimized our risk (aka expected loss) in the Bayesian Decision Theoretic sense, over some loss function $\sL(h_{\text{true}}, h, m, t)$ that takes as parameters the true state of the world $h_{\text{true}}$, our guess $h$, and the money $m$ and time $t$ spent arriving at our guess.  

\subsection{The Queries}

We receive a stream of queries.
 The world, in the context of a query, is composed of vector of $n$ random variables, $[X_1 \ldots X_n] \in \sX$, where $\sX$ is the input domain.
 Each variable $X_i$ is categorical with domain $D_i = [1 \ldots K_i]$, so the size of domain $|D_i| = K_i$, and $K_i$ need not equal $K_j$ if $i \neq j$.

Each query asks us to distinguish between a number of possible \textit{hypotheses} $h \in \sH$, where $\sH$ is the space of all possible assignments to the $X_i$'s, so $\sH = D_1 \times \ldots \times D_n$.
 Each hypothesis $h$ corresponds to a unique assignment to a set of query variables $[X_1 = x_1, \ldots, X_n = x_n]$. 

Each query is further given a markov network to describe the features of the data relating the $X_i$'s.
 The markov net consists of a set of features over the data $f \in \sF$.
 Here $\sF$ is the space of all possible features extracted over families of variables $X_i$.

The likelihood of a single joint assignment $\bX = \bx$, where $\bx$ corresponds to some $h \in \sH$, is the normalized soft max of the inner product of the weights of our model $w$ (not given in the query, but learned over time) and the feature values at $\bx$.


\[P(\bX = \bx) = \frac{1}{Z}exp(\sum_k w_k^Tf_k(\bx_{\{k\}}))\]
\[Z = \sum_{x\in\sX} exp(\sum_k w_k^Tf_k(\bx_{\{k\}}))\]

To provide compact representation for the rest of the paper, we will say that each query is provided a \textit{graph}, which is a tuple of variables and feature functions $G = ([X_1, \ldots, X_n], [f_1, \ldots, f_m])$.


\subsection{The Crowd Workers}

Our learner will be allowed to query humans for additional certainty about individual random variables (nodes) $X_i$.
 Humans will exist in a pool $\sP$.
 An individual human $o_i \in \sP$ (for ``oracle'', using the loosest possible definition of oracle) has a model for expected behavior.
Our learner will be allowed to query humans for additional certainty about individual random variables (nodes) $X_i$.
 Humans will exist in a pool $\sP$.
 An individual human $o_i \in \sP$ (for ``oracle'', using the loosest possible definition of oracle) has a model for expected behavior. Specifically, we model the $o_i$ expected delay to respond to a question, and the error function (i.e. response given the true state of the world $h_{\text{true}}$).

For this work we use a simplified model of error, shared uniformly across crowd workers.
 Previous work \cite{yan2011active} \cite{donmez2008proactive} \cite{golovin2010near} has shown that in an offline setting treating oracles uniformly leads to a loss, but in practice our pool is churning so quickly that we don't have time to learn accurate distinctions between workers.
 We leave a solution to this problem to future work.

When asked about variable $X_j$, human error is modeled as correct (returns $h_{\text{true}}(X_j)$) with probability $1-\epsilon$, and chosen uniformly at random from $D_j$ with probability $\epsilon$.
 We use the notation $Q(o_i, X_j)$ to denote the response from asking human $i$ about random variable $j$.

\begin{equation}
    Q(o_i, X_j) \sim
    \begin{cases}
       \text{uniformly drawn from } D_j = \{1 \ldots K_j\}, & \text{with probability}\ \epsilon \\
      h_{\text{true}}(X_j), & \text{otherwise}
    \end{cases}
 \end{equation}
 
This answer arrives after a delay.
 We model the amount of time the worker takes to answer with a gaussian, parameterized by $\mu$ and $\sigma$.
 We use the notation $\sD(o_i, X_j)$ to denote the delay in response when asking human $i$ about random variable $j$.

\[\sD(o_i, X_j) \sim \sN(\mu, \sigma^2)\]

This is an imperfect model, since it assigns some mass to a negative response time, which is impossible, but given a large $\mu$ and relatively small $\sigma$ the mass assigned to $\sD(o_i, X_j) < 0$ is negligible, and it otherwise accurately reflects human response times.

\subsection{The Loss Function}

Our goal is to choose our querying strategy to minimize an arbitrary loss function:
\[\sL(h_{\text{true}}, h, m, t)\]
$\sL$ takes as parameters the true state of the world $h_{\text{true}}$, our guess $h$, and the money $m$ and time $t$ spent arriving at our guess.
 By tweaking $\sL$ practitioners applying our framework can move to arbitrary points on the cost-delay-accuracy tradeoff surface.
 We deliberately make no assumptions about the form of $\sL$ while developing theory, and demonstrate several choices in our experiments.




% Fun story description!
%Eve has spent some time thinking about her labeling task and has constructed a model $p_{\theta}( \by \given \bx )$, the specifics of which are irrelevant at this point.
%At some time $t$, a user of her system requests a data point $\bx\oft{t}$ to be labeled. The user expects a particular labeling $\ys$, though Eve and her system are not aware of this.
%Instead, Eve returns the model's best prediction: $\byt = \argmax_{\by} p_{\theta\oft{t}}(\by \given \bx\oft{t})$. 
%The user upon seeing this labeling implicitly (or explicitly) penalizes Eve for any inaccuracies using the loss function $\ell(\byt, \bys)$, either by not paying for her service or not using it in the future. 
%Eve unfortunately never gets to see this feedback.

